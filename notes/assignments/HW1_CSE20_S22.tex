\documentclass[10pt,letterpaper,unboxed,cm]{article}
\usepackage[margin=1in]{geometry}
\usepackage{graphicx}
\usepackage{enumerate, comment}
\usepackage{adjustbox}

\newcommand{\st}{~\mid~}
\newcommand{\ind}{$~~~~~$}
\usepackage[noend]{algpseudocode}
\usepackage{algorithm,algorithmicx,amsmath,amssymb}


\newcommand{\A}[0]{\texttt{A}}
\newcommand{\C}[0]{\texttt{C}}
\newcommand{\G}[0]{\texttt{G}}
\newcommand{\U}[0]{\texttt{U}}

\begin{document}


\hfill{CSE 20 Spring 2022}

\hfill{Homework 1}

\hfill{Due date: Wednesday, Apr 9, 2022 at 11:59pm}



{\bf In this assignment,}

You will practice reading and
applying definitions to get comfortable working with mathematical language. As
a result, you can expect to spend more time reading the questions and looking
up notation than doing calculations.


{\bf For all HW assignments:}

Weekly homework may be done individually or in groups of up to 4 students. 
You may switch HW partners for different HW assignments. 
The lowest HW score will not be included in your overall HW average. 
Please ensure your name(s) and PID(s) are clearly visible on the first page of your homework submission.

All submitted homework for this class must be typed. 
Diagrams may be hand-drawn and scanned and included in the typed document. 
You can use a word processing editor if you like (Microsoft Word, Open Office, Notepad, Vim, Google Docs, etc.) 
but you might find it useful to take this opportunity to learn LaTeX. 
LaTeX is a markup language used widely in computer science and mathematics. 
The homework assignments are typed using LaTeX and you can use the source files 
as templates for typesetting your solutions\footnote{To use this template, copy the source file (extension \texttt{.tex}) 
to your working directory or upload to Overleaf.}.

{\bf Integrity reminders}
\begin{itemize}
\item Problems should be solved together, not divided up between the partners. The homework is
designed to give you practice with the main concepts and techniques of the course, 
while getting to know and learn from your classmates.
\item You may not collaborate on homework with anyone other than your group members.
You may ask questions about the homework in office hours (of the instructor, TAs, and/or tutors) and 
on Piazza (as private notes viewable only to the Instructors).  
You \emph{cannot} use any online resources about the course content other than the class material 
from this quarter -- this is primarily to ensure that we all use consistent notation and
definitions we will use this quarter.
\item Do not share written solutions or partial solutions for homework with 
other students in the class who are not in your group. Doing so would dilute their learning 
experience and detract from their success in the class.
\end{itemize}



\textbf{In this class, unless the instructions explicitly say otherwise, you are required to justify all your answers.}

 
\begin{enumerate}
\item (12 points)
For each set, list out the elements in roster notation: (you do not need to show your work.)
\begin{enumerate}
\item
$\{\A\A,\A,\C\}\circ\{\A\A,\A\}$
\item
$\{\A\A,\A,\C\}\times\{\A\A,\A\}$
\item
$\{(\A,\C),\A\}\times\{\C,(\A,\U)\}$
\item
$\{x\in \mathbb{N}~|~9 \leq x^2< 50\}$
\item
$\{x^2~|~x\in \mathbb{N} ~and~ 9\leq x^2<50\}$
\item
$\{\lambda,00\}\circ\{1,11,111\}$
\end{enumerate}

\item (10 points)
Recall the recursively defined set $S$ of all RNA strands over the bases in the set $B=\{\A,\C,\G,\U\}$:

\begin{center}
Basis Step:    $\A\in S, \C\in S, \U\in S, \G\in S$\\
Recursive Step: If $s\in S$ and $b\in B$ then $sb\in S$
\end{center}

{\bf Exercise:} Recursively define a function $AminusC$ from the set $S$ to the set $\mathbb{N}$ that computes the number of $\A$s minus the number of $\C$s.  For example, $AminusC(AGGUCUC)=-1$

Your function needs to follow the format of a recursive function.  Namely,
\begin{center}
Basis Step:    $<your \ basis \ step>$\\
Recursive Step: $<your \ recursive \ step>$
\end{center}


\item (15 points)
Recall the way we encoded Netflix ratings in class. Let's say that Netflix only has 4 titles:
\begin{center}
 Bridgerton, Squid Game, Ozark, Tiger King.
\end{center}
Then for example, a user with the 4-tuple: $(-1,1,-1,0)$ is a user who does not like Bridgerton or Ozark, likes Squid Game and they are indifferent to Tiger King. (maybe they haven't seen it.)

For example:

Suppose Netflix has 6 Users:

\begin{tabular}{cccccccc}
User 1: &(1,&0,&-1,&0)\\
User 2: &(-1,&-1,&1,&-1)\\
User 3: &(1,&1,&0,&1)\\
User 4: &(0,&-1,&0,&0)\\
User 5: &(-1,&-1,&1,&1)\\
User 6: &(-1,&-1,&-1,&-1)
\end{tabular}

In class we learned about a way to compare two users by taking the \emph{Euclidean distance} of their $4$-tuples.

We can apply the same logic to compare two titles.



\begin{quote}
{\bf Exercise 1:} (4 points) For each title, create a $6$-tuple where the first entry is the rating of User 1, the second entry is the rating of User 2 and so on.
\end{quote}

\begin{quote}
{\bf Exercise 2:} (4 points) Which pair of Titles are ``closest" and which pair of Titles are ``furthest" from each other using the $dist$ metric? (show all 6 calculations.)
\end{quote}

Another way to compare two titles is by using the \emph{dot product} of their $6$-tuples. The dot product is computed by multiplying corresponding entries and adding up the products. For example:

$$dot((-1,1,-1,0,1,0),(1,0,0,1,-1,1))=$$
$$(-1)(1)+(1)(0)+(-1)(0)+(0)(1)+(1)(-1)+(0)(1)=-2$$


\begin{quote}
{\bf Exercise 3:} (3 points)

Compute the dot product of Squid Game and Tiger King.

\end{quote}

We can use the dot product to define another measure of how similar two titles are. It is called the cosine similarity and for two $n$-tuples $v,w$ it is defined as:

$$cossim(v,w) = \frac{dot(v,w)}{\sqrt{dot(v,v})\sqrt{dot(w,w)}}$$

For example, the cossim of $(-1,1,-1,0,1,0),(1,0,0,1,-1,1)$ is:

$$cossim((-1,1,-1,0,1,0),(1,0,0,1,-1,1)) =$$
$$ \frac{dot((-1,1,-1,0,1,0),(1,0,0,1,-1,1))}{\sqrt{dot((-1,1,-1,0,1,0),(-1,1,-1,0,1,0)})\sqrt{dot((1,0,0,1,-1,1),(1,0,0,1,-1,1))}}=$$
$$\frac{-2}{\sqrt{4}\sqrt{4}}=-1/2$$


\begin{quote}
{\bf Exercise 4:} (3 points)

Compute the cosine similarity of Squid Game and Tiger King (you can use your calculation from the previous exercise.)
\end{quote}


The distance function and the cosine similarity function are two different ways to measure how similar two user ratings are.

\begin{quote}
{\bf Exercise 5:} (for fair effort completeness:) (1 point)

In your own words, what are the differences and similarities of these two functions? Which one do you like better for the task of comparing two titles and why?
\end{quote}

\item (10 points)
Color in computer is often represented as a 3-tuple $(r,g,b)$ where $r$ represents the red component, $g$ represents the green component and $b$ represents the blue component. Each component $r,g,b$ must be an integer between 0 and 255 (inclusive.)

For example, a nice lavender color is $(199,176,252)$. White is $(255,255,255)$ and black is $(0,0,0)$.

Let $A=\{x\in\mathbb{N}~|~0\leq x\leq 255\}$ be the range of each of the rgb components.


Let's say that you have 8 different paint colors:

\begin{tabular}{ccccc}
A&(103,&221,&21)\\
B&(28&,102,&71)\\
C&(164,&110,&70)\\
D&(67,&12,&119)\\
E&(187,&70,&124)\\
F&(148,&79,&150)\\
G&(226,&230,&42)\\
H&(27,&188,&202)
\end{tabular}

\includegraphics[scale=0.4]{colors}

Think about how you would describe a measure of ``closeness" of two colors. Out of the 8 colors, using your description would you say that A and B should be considered the ``closest" pair? What is a function you can use to quantify this idea?


Let's consider the cossim formula for two colors $v,w$:

$$cossim(v,w) = \frac{dot(v,w)}{\sqrt{dot(v,v})\sqrt{dot(w,w)}}$$
\begin{quote}

{\bf Exercise 1:} (4 points)

It looks like the two green colors (A and B) should be closest. Compute the cossim of these two colors.

{\bf Exercise 2:} (4 points)

Consider the two purply colors D and F. Compute the cossim of these two colors.

{\bf Exercise 3:}  (for fair effort completeness) (1 point)

How do you interpret the results of Exercise 1 and Exercise 2? according to the results, would you say the two greens are closer or the two purples are closer and why?

{\bf Exercise 4:} (for fair effort completeness) (1 point)

Is the Euclidean distance function a better or worse way to compare colors? Why or why not?
\end{quote}



 
\end{enumerate}



\end{document}